% Options for packages loaded elsewhere
\PassOptionsToPackage{unicode}{hyperref}
\PassOptionsToPackage{hyphens}{url}
%
\documentclass[
]{article}
\usepackage{amsmath,amssymb}
\usepackage{lmodern}
\usepackage{ifxetex,ifluatex}
\ifnum 0\ifxetex 1\fi\ifluatex 1\fi=0 % if pdftex
  \usepackage[T1]{fontenc}
  \usepackage[utf8]{inputenc}
  \usepackage{textcomp} % provide euro and other symbols
\else % if luatex or xetex
  \usepackage{unicode-math}
  \defaultfontfeatures{Scale=MatchLowercase}
  \defaultfontfeatures[\rmfamily]{Ligatures=TeX,Scale=1}
\fi
% Use upquote if available, for straight quotes in verbatim environments
\IfFileExists{upquote.sty}{\usepackage{upquote}}{}
\IfFileExists{microtype.sty}{% use microtype if available
  \usepackage[]{microtype}
  \UseMicrotypeSet[protrusion]{basicmath} % disable protrusion for tt fonts
}{}
\makeatletter
\@ifundefined{KOMAClassName}{% if non-KOMA class
  \IfFileExists{parskip.sty}{%
    \usepackage{parskip}
  }{% else
    \setlength{\parindent}{0pt}
    \setlength{\parskip}{6pt plus 2pt minus 1pt}}
}{% if KOMA class
  \KOMAoptions{parskip=half}}
\makeatother
\usepackage{xcolor}
\IfFileExists{xurl.sty}{\usepackage{xurl}}{} % add URL line breaks if available
\IfFileExists{bookmark.sty}{\usepackage{bookmark}}{\usepackage{hyperref}}
\hypersetup{
  pdftitle={Adjusting Manual Rates to Own Experience: Comparing the Credibility Approach to Machine Learning},
  pdfauthor={Giorgio A. Spedicato, Ph.D FCAS FSA CSPA, Christophe Dutang, Ph.D, Quentin Guibert, Ph.D},
  hidelinks,
  pdfcreator={LaTeX via pandoc}}
\urlstyle{same} % disable monospaced font for URLs
\usepackage[margin=1in]{geometry}
\usepackage{longtable,booktabs,array}
\usepackage{calc} % for calculating minipage widths
% Correct order of tables after \paragraph or \subparagraph
\usepackage{etoolbox}
\makeatletter
\patchcmd\longtable{\par}{\if@noskipsec\mbox{}\fi\par}{}{}
\makeatother
% Allow footnotes in longtable head/foot
\IfFileExists{footnotehyper.sty}{\usepackage{footnotehyper}}{\usepackage{footnote}}
\makesavenoteenv{longtable}
\usepackage{graphicx}
\makeatletter
\def\maxwidth{\ifdim\Gin@nat@width>\linewidth\linewidth\else\Gin@nat@width\fi}
\def\maxheight{\ifdim\Gin@nat@height>\textheight\textheight\else\Gin@nat@height\fi}
\makeatother
% Scale images if necessary, so that they will not overflow the page
% margins by default, and it is still possible to overwrite the defaults
% using explicit options in \includegraphics[width, height, ...]{}
\setkeys{Gin}{width=\maxwidth,height=\maxheight,keepaspectratio}
% Set default figure placement to htbp
\makeatletter
\def\fps@figure{htbp}
\makeatother
\usepackage[normalem]{ulem}
% Avoid problems with \sout in headers with hyperref
\pdfstringdefDisableCommands{\renewcommand{\sout}{}}
\setlength{\emergencystretch}{3em} % prevent overfull lines
\providecommand{\tightlist}{%
  \setlength{\itemsep}{0pt}\setlength{\parskip}{0pt}}
\setcounter{secnumdepth}{-\maxdimen} % remove section numbering
\usepackage{draftwatermark}
\ifluatex
  \usepackage{selnolig}  % disable illegal ligatures
\fi
\newlength{\cslhangindent}
\setlength{\cslhangindent}{1.5em}
\newlength{\csllabelwidth}
\setlength{\csllabelwidth}{3em}
\newenvironment{CSLReferences}[2] % #1 hanging-ident, #2 entry spacing
 {% don't indent paragraphs
  \setlength{\parindent}{0pt}
  % turn on hanging indent if param 1 is 1
  \ifodd #1 \everypar{\setlength{\hangindent}{\cslhangindent}}\ignorespaces\fi
  % set entry spacing
  \ifnum #2 > 0
  \setlength{\parskip}{#2\baselineskip}
  \fi
 }%
 {}
\usepackage{calc}
\newcommand{\CSLBlock}[1]{#1\hfill\break}
\newcommand{\CSLLeftMargin}[1]{\parbox[t]{\csllabelwidth}{#1}}
\newcommand{\CSLRightInline}[1]{\parbox[t]{\linewidth - \csllabelwidth}{#1}\break}
\newcommand{\CSLIndent}[1]{\hspace{\cslhangindent}#1}

\title{Adjusting Manual Rates to Own Experience: Comparing the
Credibility Approach to Machine Learning}
\author{Giorgio A. Spedicato, Ph.D FCAS FSA CSPA, Christophe Dutang,
Ph.D, Quentin Guibert, Ph.D}
\date{04 luglio, 2021}

\begin{document}
\maketitle

\hypertarget{introduction}{%
\section{Introduction}\label{introduction}}

The use of market data as an aid for setting own company rates has been
a common practice in the Insurance Industry. External data, as provided
by Insurance Rating Bureaus, \textcolor{blue}{reinsurers} or Advisory
Organizations, may supplement internal company's ones that may be scarce
\textcolor{blue}{or} unreliable because of a non-representative and/or a
too short history, or, non-existing at all, e.g.~when entering in a new
business lines or territory. The importance of external data to both
support adequate rates that preserve company solvency and in easing the
entrance of new players has been historically recognized by regulator,
e.g.~granting a partial antitrust -- law exception in the US
jurisdiction (Danzon 1983).

When the \textcolor{blue}{insurer} takes into account its own experience
in order to enhance the credibility of its rates, \textcolor{blue}{he}
need to \textcolor{blue}{benchmark} its portfolio experience
\textcolor{blue}{compared} the market one. The actuarial profession
traditionally used techniques based on Bayesian statistics and
non-parametric credibility to optimally combine the market and insurer's
portfolio experience in the technical rates.
\textcolor{blue}{From this point of view, the contribution of market data makes it possible to satisfy the two classic approaches addressed by the theory of credibility, the so-called limited fluctuation credibility theory and greatest accuracy credibility theory [@norberg2004credibility]. The former refers to need of incorporate individual experience into the rate calculation in order to stabilize the level of individual rates. The second approach corresponds to the application of modern credibility theory and consists in combining both individual and collective experience to predict individual rates by mean square error minimization. Credibility theory is extensivily used in non-life insurance and early models were not based on policyholders’ rate-making variables, see e.g. @buhlmann2006course for a review. Yet some advanced regression credibility models have been proposed in the actuarial literature, such as the so-called Hachemeister model [@hachemeister1975credibility].}
On the contrary, rates based on Generalized Linear Models (GLM), the
current gold standard in personal rates pricing (Goldburd, Khare, and
Tevet 2016), are only based on the impact of ratemaking factors, giving
no credit to the individual policy experience. Nevertheless, mixed
effects GLMs allow to incorporate policyholders' experience within GLM
tariff structure ({``{Bayesian credibility for GLMs}''} 2018; Antonio
and Beirlant 2007) but they are not widespread used.
\textcolor{blue}{All these regressions approaches are enable to incorporate individual risk profile covariates into a credibility model. The structure of insurance data, notably the distinction between own experience and market experience, is dealed with the use of the hierarchic credibility model of @buhlmann1970glaubwurdigkeit.}

\textcolor{blue}{Credibility theory is also largely used in life insurance applications for modeling mortality risks. A first attempt for stabilizing mortality rates by combining the mortality data of the small population with the average mortality of the neighboring populations is proposed by @ahcanforecasting2014. Regarding this issue of limited mortality data (small population or short historical period of observations), @libayesian2018 introduce a Bayesian non-parametric model for benchmarking a small population compared to a reference population, and @bozikascredible2019 focus on a credible regression framework to efficiently forecast populations with a short-base-period. In order to improve mortality forecasting, some recent contributions have been done in the literature for combining usual mortality models, such as the [@leemodeling1992] model and Bühlmann credibility theory, see @tsai2017incorporating, @tsaimultidimensional2019 and @tsaiincorporating2020 among other.}

The recent widespread/massive usage of Machine Learning has provided
many more techniques to the practitioner actuaries'
\textcolor{blue}{toolkit, see @blierwongmachine2021 for a review in non-life insurance}.
\textcolor{red}{TO DO : summarize the main contributions of ML for pricing and reserving.}.
\textcolor{blue}{Recently, @diaoregression2019 combine the use of credibility and regression tree models}.
\textcolor{blue}{In these publications, the ultimate goal of the use of Machine Learning is to improve the usual regression setup in actuarial science based on the GLM. However, these techniques, such as  the Gradient Boosting Models (GBM) and the Deep Learning (DL), can also be used in a manner that permits to “transfer” what the model has \textcolor{blue}{learned} on a much bigger data set (as the market data, MKT) to a smaller set (the portfolio data of the company, CPN).}
``Transfer learning'' (henceforth TFL) is typically in computer vision
DL modes to fine tune standard architecture on specific recognition
tasks. An ``initial score'' can be provided to GBMs to take into account
the known effect of exposure or a-priori modeled estimate before
``boosting'' the prediction.
\textcolor{blue}{In this paper, our aim is to exploit the advantages of ML for easily handling complex non–linear relationship compared to standard credibility or GLMs based approaches to assess the policyholders’ risk more precisely. Hence, our work contrast with}
traditional methods to ML ones in the task of blending market data to
individual portfolio experience.

\textcolor{blue}{The rest of this paper is organized as follows.} After
a brief business and methodological introduction, we apply
\textcolor{blue}{in Section XX} on a (properly anonymized) data set
comprised by both market and own portfolio experience relative to a
European country non -- life business line. The final comparisons will
consider not only on the predictive performance, but also in the ease of
practical application in term of computational request, ease of
understanding and interpretability of the results.

\hypertarget{overview-of-the-use-of-external-experience-in-own-business-in-the-insurance-industry}{%
\section{Overview of the use of external experience in own business in
the insurance
industry}\label{overview-of-the-use-of-external-experience-in-own-business-in-the-insurance-industry}}

To be discussed:

\begin{itemize}
\tightlist
\item
  historical practice in the US; current use of insurance bureau;
\item
  use of external experience, credibility
\item
  Role of reinsurance as providing advice and support in pricing.
\item
  \textcolor{red}{deduce the portfolio experience of the market by extracted the rates published by competitors.}
\end{itemize}

\textcolor{red}{Apart from the references cited in introduction, I have no particular knowledge on this subjet. In France, it seems that this practise is less developed. I suggest to briefly describe these practises and then to describe more preciseely how our data set is organized. In order to fit within the framework of credibility theory, it is important to explain that:}

\begin{itemize}
\tightlist
\item
  \textcolor{red}{the market data set is much bigger than the company data set and is thus almost equal to the collective premium when comibining the data set.}
\item
  \textcolor{red}{the process on which we focus is recursive and consists in predicting rates of year $n$ based both on the market and the company experiences accumulated on the claim history from year $0$ to $n-1$.}
\end{itemize}

\hypertarget{credibility-based-models-with-covariates}{%
\section{\texorpdfstring{\textcolor{blue}{Hierarchical}
credibility-based models with
covariates}{ credibility-based models with covariates}}\label{credibility-based-models-with-covariates}}

\emph{TODO}

\begin{itemize}
\tightlist
\item
  \textcolor{red}{notation and assumptions.}
\item
  \textcolor{red}{insert the hierarchical tree structure.}
\item
  \textcolor{red}{explain how integrate market data.}
\item
  \textcolor{red}{briefy recall the estimators related the hierarchic Bülhmann credibility theory and introduce the use of covariates as in the hachemeister model.}
\end{itemize}

\hypertarget{modeling-approach}{%
\section{\texorpdfstring{Modeling approach
\textcolor{blue}{with machine learning}}{Modeling approach }}\label{modeling-approach}}

\textcolor{blue}{This} research aims to compare the predictive power of
traditional \textcolor{blue}{credibility} and machine learning methods
that use an initial estimate of loss costs, e.g.~from
\textcolor{blue}{MKT} experience, to predict those of a smaller portion
(the CPN one) in a subsequent period (the test set). Therefore,
\textcolor{blue}{following the idea of the greatest accuracy credibility theory, our}
modeling process aims to predict the losses on the last available year
(the test set) by training models on the experience of the previous
years, eventually split into a train and validation test.
\textbackslash textcolor\{blue\}\{Then, we focus on models that permits
to use an initial estimate of losses performed on another set (transfer
learning). While the paper explores the use of such approach applying ML
methods, traditional GLM may be used as well. For instantce, under a
log-linear regression framework and initial log-estimate of either the
frequency, the severity of the pure - premium may be set as an offset
for a subsequent model (Yan et al. 2009).\}

The empirical data available for the study regards a risk for which year
to year loss cost may materially fluctuate due to external condition
(systematic variability) much more than the portolios' risks
heterogeneity composition. At this regards, the performance assessment
has considered not only the discrepancy between the actual and predicted
losses, but the ability of the model to rank risks
\textcolor{red}{What do you mean by "rank risks" and how ML can achieve this purpose}.

The losses are the number of damaged units while the exposure are the
number of insured units. Therefore\textcolor{blue}{,} only the frequency
component has been modeling, choosing either a
\textcolor{blue}{B}inomial or a Poisson loss function. Henceforth losses
in this paper shall be considered as synonym of claim number.

\sout{\#\# Machine learning techniques}
\textcolor{red}{I suggest to  move this paragraph into the introduction}
ML methods have been acquiring increasing attention by actuarial
practitioners especially. Beginning from the analysis of policyholders'
behavior (Spedicato, Dutang, and Petrini 2018), several applications
have sprung also for risk pricing. An application of boosting techniques
to estimate the frequency and the severity of an MTPL dataset
\textcolor{red}{define MTPL} can be found in
\textcolor{blue}{@noll2020case, while @schelldorfer2019nesting present a joint model that boosts GLM performance using Deep Learning}.

ML methods used in insurance pricing are strongly non - linear and are
able to automatically find interactions among ratemaking factors and
exclude non relevant features. In particular two techniques are
acquiring widespread importance: Boosting and Deep learning. Both
techniques allow the use of an initial estimate of loss / exposure to
risk \textcolor{blue}{to train the model on last observations.}
\sout{tho be feed to the model to be fine tuned for the current
dataset}.

\sout{\#\# Brief overview}

\hypertarget{boosting-techniques}{%
\subsection{Boosting techniques}\label{boosting-techniques}}

The boosting approach \textcolor{blue}{[@friedman2001greedy]} can be
synthesized by the following formula:

\[F_t\left(x\right)=F_{t-1}\left(x\right)+\eta*h_t\left(x\right)\] that
is, the prediction at the \(t\)-it step is given by the contribution, to
the prediction of the previous step, of a weak predictor
\(h_t\left(x\right)\), properly weighted by a learning factor \(\eta\),
being \(x\) the covariate vector. The most common choice for the weak
predictor \(h_t\left(x\right)\) lies in the classification and
regression trees family, from which the Gradient Boosted Tree (GBT)
models. It can be shown that ``boosting'' weak predictors lead to very
strong predictive models (Elith, Leathwick, and Hastie 2008). Almost all
winning solutions of data science competitions held by Kaggle are at
least partially based on XGBoost (Chen and Guestrin 2016), the most
famous GBT model. More recent and interesting alternatives to be tested
are: LightGbm (Ke et al. 2017), which is particularly renowned for its
speed, and Catboost (Prokhorenkova et al. 2017), which has introduced an
efficient solution for handling categorical data.

\textcolor{red}{I suggest to describe the LightGbm method (notations, loss function and the algorithm). Maybe it could be interesting to explain why you prefer LGBM on the selected data set compared to other boosting methods. Do you have tested other approaches before and select the LGBM as the winner?}

A set of hyperparameter defines a boosted model and even more define a
GBT one. The core hyperparameters that influence the boosting part are
the number of models (trees), \(t=1,2,\ldots, T\) (typically between 100
and 1000) and the learning rate \(\eta\), whose typical values lies
between 0.2 and 0.001. \(h_t\left(x\right)\) can be, when it belongs to
the CART family, the maximum depth, the minimum number of observation in
final leafs, the fraction of observation (rows or columns) that are
considered when growing each tree. The optimal combination of
hyperparameters is learn using either a grid search approach or a more
refined one (e.g.~bayesian optimization).

When applied to claim frequency prediction, they are fit to optimize a
Poisson log-loss function. In addition, to handle uneven risk exposure,
the log - measure of exposure risk is given (in log scale) as an
init-score (\(F_t\left(x\right)\)) to initialize the learning process.
The init-score (or base margin) in the boosting approach has the same
role of the traditional GLM offset term (Goldburd, Khare, and Tevet
2016).

\hypertarget{deep-learning}{%
\subsection{Deep Learning}\label{deep-learning}}

An artificial network is a mathematical structure that applies a non
linear function to a linear combination of inputs, say
\(\phi\left(\bar x_i^T \times \bar w+\beta\right)\), being \(\bar w\)
and \(\beta\) the weights and bias respectively. A neural network
consists in one \textcolor{blue}{of} more layer of interconnected
neurons, that receives a (possibly multivariate) input set and retrieves
and output set \textcolor{red}{Cite a general reference for NN}. Modern
Deep Neural Networks are constructed by many (deep) layers of neurons.
Deep Learning has been knowing a hype in interest for a decade, thanks
to the availability of huge amount of data, computing power (in
particular GPU computing) and the development of newer approaches to
reduce the overfitting that had halted the widespread adoption of such
techniques in previous decades (Goodfellow, Bengio, and Courville 2016).
Different architectures has reached state of the art performances in
many fields; e.g.~convolutionary neural networks achieved top
performance in computer vision (e.g.~image classification and object
detection) \textcolor{red}{Cite a general reference}, while recurrent
neural networks
\textcolor{blue}{, see e.g. @hochreiter1997long for Long Short Term Memory ones},
provides excellent results in Natural language processing tasks like
sequence-to-sequence modeling (translation) and text classification
(sentiment analysis).
\textcolor{blue}{For applications in actuarial science, we refer to the recent review of @blierwongmachine2021, and to the the work of 
@richmanai2021 and @richmanai20211 for deep neural network.}

Simpler structure are needed for a claim frequency regression, the
multi-layer perception (MLP) architectures that basically consist in
stacked simple neurons layers, from the input one to the single output
cell one. This structure is dealt to handle the relation between the
relation between the ratemaking factors and the frequency (the
structural part). To handle the different exposures, the proposed
architecture is based on the solution presented by (Ferrario, Noll, and
Wuthrich 2020; Schelldorfer and Wuthrich 2019). A separate branch
collects the exposure, applies a log-transformation, then this exposure
is added in a specific layer just before the final one (that has a
dimension of one).

Training a DL model consists in providing batches of data to the
network, evaluate the loss performance and updating the weights in the
direction that minimize the training (backpropagation). The whole data
set is provided to the fitting algorithms many times (epochs) split in
batch. One of the common practice to avoid overfitting is to use a
validation set where the loss is scored at each epoch. When it starts to
systematically diverge, the training process is stopped (early
stopping).

\textcolor{red}{I suggest to described here with more the NN structure and the activation function used in our application, as well as the algorithm used. }

\emph{TODO}

\hypertarget{section}{%
\section{\texorpdfstring{\textcolor{blue}{Numerical application}}{}}\label{section}}

\textcolor{blue}{In this study, the analysis is performed on two real and anonymized data sets},
the CMP and MKT, preprocessed and split into train, validation and test
set as previously discussed. Then, the models
\textcolor{blue}{is fitted} on the train set and the predictive
performance \textcolor{blue}{is} assessed on the test set. The
validation set was used in DL and BST models to avoid overfitting.
Finally, the models are compared in terms of predictive accuracy, using
the (The actual / predicted ratio) and risk classification performance,
using the Normalized Gini index (Frees, Meyers, and Cummings 2014). The
latter index has become quite popular in the actuarial academia and
\textcolor{blue}{practitioners} to compare competing risk models.

\textcolor{red}{Describe the computer environnement and cite solfwares used.}

\hypertarget{the-structure-of-the-dataset}{%
\subsection{The structure of the
dataset}\label{the-structure-of-the-dataset}}

Two (anonymized) dataset were provided, one for the marketwide
(\texttt{"mkt\_anonymized\_data.csv"}) and one for the company
(\texttt{"mkt\_anonumized\_data.csv"}), henceforth MKT and CMP datasets.
\textcolor{blue}{These} datasets share the same structure, as each
company provides its data in the same format to the Pool, that
aggregates individual filings into a marketwide file, that is provided
back to the companies. The dataset contains
\textcolor{blue}{the exposures and claim numbers,} aggregated by some
\textcolor{blue}{categorical} variables. Variable names, levels and
numeric variable distribution have been masked and anonymized for
privacy and confidentiality purposes.

\textbf{maybe a graphic could be done?}
\textcolor{red}{yes: display the schematic of the process at training/testing stages.}

\textcolor{red}{Would it be possible to display summary statistics for describing the volume of data, the effect of categorical variables and the distribution of continuous variables? Maybe, it could be interesting to compare the contribution of the company into the market data set in terms of claim numbers.}

The following variables are contained in the provided data set:
\textcolor{red}{Make a table, and if possible with some summary statistics (mean, median, Q1, Q3, sd, number of levels, ...)}

\begin{itemize}
\tightlist
\item
  \emph{exposure}: the insurance exposure measures by classification
  group, on which the rate is filled (aggregated outcomes)
  \textcolor{red}{Define "classification group"};
\item
  \emph{claims}: the number of claims by classification group
  (aggregated outcomes);
\item
  \emph{ID}: unique row number (helper variable);
  \textcolor{red}{Could be removed ?};
\item
  \emph{zone\_id}: territory (aggregating variable);
\item
  \emph{year}: filing year (aggregating variable);
\item
  \emph{group}: random partition of the dataset into train, valid and
  test set.
\item
  \emph{cat1}: categorical variable 1, available in the original file
  (aggregating
  variable);\textcolor{red}{suggestion : renames 'cat1' and 'cat2' as 'risk class 1' and 'risk class 2'}
\item
  \emph{cat2}: categorical variable 2, available in the original file
  (aggregating variable);
\item
  \emph{cat3}: categorical variable 3, available in the original file
  (aggregating variable);
\item
  \emph{cat4-cat8}: categorical variables related to the territory
  (joined to the original file by zone\_id);
\item
  \emph{cont1-cont12}: numeric variables related to the territory
  (joined to the original file by zone\_id);
\end{itemize}

Categorical and continuous variables have been anonymized by label
encoding and scaling (calibrated on market data). In addition, the last
available year (2008) has been used as test set, while data from
precedent years have been randomly split between train and validation
sets on a 80/20 basis. Market data is available for 11 years, while
company data for the last five one. Also, the number of exposures is
widely dependent by the cat1 variable.

\hypertarget{ml-techniques}{%
\subsection{\texorpdfstring{\sout{ML techniques}
\textcolor{blue}{Implementation details}}{ML techniques }}\label{ml-techniques}}

\hypertarget{boosting-approach}{%
\subsubsection{Boosting approach}\label{boosting-approach}}

The \textcolor{blue}{lightgbm} model has been used to apply boosted
trees on the provided data sets, minimizing Poisson deviance. As for
most modern ML methods, a \textcolor{blue}{lightgbm} model is fully
defined by a set of many \textcolor{blue}{hyperparameters} for which
default values may not be optimal for the given data and there is no
closed formula to identify the best combination for the given data.

Therefore an
\textcolor{blue}{hyperparameter optimization step is performed}. For
each hyperparameter a range of variation \textcolor{blue}{is} set, then
a 100-run trial was performed using a Bayesian Optimization approach
performed by the \textbackslash textcolor\{blue\}\{\emph{hyperopt}
\textbf{Python} library\} (Bergstra, Yamins, and Cox 2013). Under the BO
approach, each subsequent iteration is \textcolor{blue}{performed}
toward the point that minimize the loss to be optimized, being the loss
distribution by hyperparameter updated each iteration using a bayesian
approach.

As suggested by boosting trees practitioners
\textcolor{red}{Add reference}, the number of boosted models
\textcolor{blue}{is} not estimated under the BO approach but determined
by early stopping. The loss \textcolor{blue}{is} scored under the
validation set and the number of trees chosen is that beyond which the
loss stop to decrease and start diverging up.

The CMP and MKT \textcolor{blue}{models use} the standard exposure (in
logarithm base) as init score. The TRF model instead uses as init score
the ``a-priori'' prediction of the MKT model on the CMP data.

\begin{itemize}
\tightlist
\item
  \textcolor{red}{Cite used packages}
\item
  \textcolor{red}{Describe the computation complexity and the run time}
\end{itemize}

\hypertarget{deep-learning-1}{%
\subsubsection{Deep Learning}\label{deep-learning-1}}

The chose DL architecture was set by several trials, based on previous
experiments and \textcolor{blue}{practitioners} architecture found in
the literature for tabular data analysis. Unfortunately, the
hyperparameter space of a DL architecture is very vaste, comprising not
only fitting level degrees of freedom (the \textcolor{blue}{optimizer},
the number of epochs, the batch size) but the whole layers'
architecture: the number of layers, the numbers of neuron within
etc\ldots{} At this regard it is common among
\textcolor{blue}{practitioners} to starts with a
\textcolor{blue}{knowing} working architecture in a
\textcolor{blue}{similar} field and perform moderate changes. While more
\textcolor{blue}{systematizing} DL architecture optimization approaches
are being developed (e.g.~the Neural Architecture Search) the use of
such techniques \textcolor{blue}{is} out of the scope of the paper.

\textcolor{red}{suggestion: gather the all the cat variables in one box and do the same with cont variables.}

\textcolor{red}{suggestion: this figure is difficult to read for me.}

\begin{figure}
\centering
\includegraphics{./materials/dl_model.png}
\caption{DL model structure}
\end{figure}

The same model shown above \textcolor{blue}{is} used for both the CMP,
MKT and TRF models. A dense layer collects the inputs, where the
categorical variables have been handled using embedding. Three hidden
layers perform the feature engineering and knowledge extraction from the
input; Dropout
layers\textcolor{blue}{is added to increase the robustness of the process}.
As anticipated in the methodological section, the exposure part is
separately handled in another branch and then merged in the final layer.

Overfitting \textcolor{blue}{is} controlled using an EarlyStopping
callback scoring the loss on the validation test and stop learning over
next epochs \textcolor{red}{cite reference} if the loss did not improve
for more than 20 epochs.

The TRF model has been build using the pre-trained weights calculated on
the market data and continuing the \textcolor{blue}{training} process on
the CPN data.

\begin{itemize}
\tightlist
\item
  \textcolor{red}{Cite used packages}
\item
  \textcolor{red}{Describe the computation complexity and the run time}
\end{itemize}

\hypertarget{credibility}{%
\subsubsection{\texorpdfstring{Credibility
\textcolor{blue}{model}}{Credibility }}\label{credibility}}

\begin{itemize}
\tightlist
\item
  \textcolor{red}{Cite used packages}
\item
  \textcolor{red}{Explain how we consider continous variables}
\item
  \textcolor{red}{Explain how we define the hierarchic structure}
\item
  \textcolor{red}{Explain data imputation in the test sample}
\end{itemize}

\hypertarget{section-1}{%
\subsubsection{\texorpdfstring{\textcolor{blue}{Assessment of performance}}{}}\label{section-1}}

\begin{itemize}
\tightlist
\item
  \textcolor{red}{Describe how are constructed the training and test samples.}
\item
  \textcolor{red}{Define the metrics considered for assesing the performance. If we directly predict the number for claims, it could be usefull to compute the RMSE, the MAE and/or the MAPE.}
\end{itemize}

\hypertarget{section-2}{%
\subsection{\texorpdfstring{\textcolor{blue}{Forecasting performance}}{}}\label{section-2}}

\textcolor{red}{For a better understanding of the results, I suggest to assess the importance of each explanatory variable.}

The table below reports the predictive performance, evaluated on the
company test set, for the DL and Bst models' families. The columns
approach indicates whether the model \textcolor{blue}{is} trained on
market-only (\textcolor{blue}{MKT} ), company-only
(\textcolor{blue}{CPN} ) or company data using a transfer learning
approach (\textcolor{blue}{TRF} ).

\begin{longtable}[]{@{}
  >{\raggedright\arraybackslash}p{(\columnwidth - 14\tabcolsep) * \real{0.06}}
  >{\raggedright\arraybackslash}p{(\columnwidth - 14\tabcolsep) * \real{0.09}}
  >{\raggedleft\arraybackslash}p{(\columnwidth - 14\tabcolsep) * \real{0.16}}
  >{\raggedleft\arraybackslash}p{(\columnwidth - 14\tabcolsep) * \real{0.22}}
  >{\raggedleft\arraybackslash}p{(\columnwidth - 14\tabcolsep) * \real{0.22}}
  >{\raggedleft\arraybackslash}p{(\columnwidth - 14\tabcolsep) * \real{0.08}}
  >{\raggedleft\arraybackslash}p{(\columnwidth - 14\tabcolsep) * \real{0.09}}
  >{\raggedleft\arraybackslash}p{(\columnwidth - 14\tabcolsep) * \real{0.09}}@{}}
\caption{Models comparison}\tabularnewline
\toprule
model & approach & normalized\_gini & actual\_predicted\_ratio &
mean\_actual\_pred\_ratio & MeanAE & MedianAE & RMSE \\
\midrule
\endfirsthead
\toprule
model & approach & normalized\_gini & actual\_predicted\_ratio &
mean\_actual\_pred\_ratio & MeanAE & MedianAE & RMSE \\
\midrule
\endhead
dl & mkt & 0.921 & 0.924 & 0.919 & 207.725 & 14.395 & 3194.502 \\
dl & cpn & 0.909 & 0.775 & 1.113 & 269.216 & 13.039 & 4976.958 \\
dl & trf & 0.925 & 0.967 & 0.944 & 201.802 & 13.291 & 3368.277 \\
bst & mkt & 0.939 & 0.975 & 0.833 & 187.390 & 10.096 & 3066.079 \\
bst & cpn & 0.924 & 0.841 & 0.860 & 249.489 & 11.450 & 4912.252 \\
bst & trf & 0.940 & 1.052 & 0.831 & 179.091 & 9.972 & 3198.866 \\
cred1 & cpn & 0.880 & 0.882 & 1.808 & 260.369 & 12.407 & 5651.256 \\
cred1 & mkt & 0.937 & 0.975 & 0.951 & 183.917 & 9.775 & 2966.997 \\
cred1 & trf & 0.898 & 1.061 & 1.369 & 233.687 & 14.345 & 5053.139 \\
\bottomrule
\end{longtable}

First, we see that the actual/predicted ratio is between 0.9 - 1.1 for
all models, but company ones - \textcolor{blue}{is the worse}. We
anticipate that as the test set considers a year different from the
train and validation \textcolor{blue}{pools}, the predictions may be
structurally biased as the insured risk depends by the year's climate
and that frequency trending is not consider in the modeling framework at
all. The transfer learning approach offers higher predictive accuracy
when measured by the NG index and the predictive performance is the
highest among all competitive approaches except for the boosting
approach. Regarding the predictive accuracy, on the other hand, and
\textbackslash textcolor\{blue\}\{especially for the DL models, we
cannot rule out that the superiority of TFR approaches holds for all
possible MLP architectures. Also, it is likely to happen that as far as
the company data increases, the advantage of the TRF approach decreases
with respect to a model trained on company data only.

\emph{TODO}

\hypertarget{conclusion}{%
\section{Conclusion}\label{conclusion}}

We presented \textcolor{blue}{an} application of TF that can be
resembled to the traditional ``credibility'' approach to transfer the
experience applied on a different, but similar, book of business to a
newer one. We saw that ML approach may provides interesting results and
may be worth to try with.

Finally, we performed our empirical analysis transferring loss
experience from an external insurance bureau to a specific company
\textcolor{blue}{portfolio.} This ``transfer of experience'' may be also
performed within the same company for example when new products,
taylored for niche books of business, are created. Initial losses
estimates may be performed on the initial product and then applied as
initial scores on the newer portfolio.

\emph{TODO}

\hypertarget{appendix}{%
\section{Appendix}\label{appendix}}

\hypertarget{code}{%
\subsection{Code}\label{code}}

The modeling has been performed using both R (R Core Team 2021) and (Van
Rossum and Drake 2009). The following files has been provided:

\begin{enumerate}
\def\labelenumi{\arabic{enumi}.}
\setcounter{enumi}{-1}
\item
  preprocess and anonymize.py: work on original data, anonymizing column
  and internal datasets and split data across train, validation and test
  set; the config\_all.py file provides ancillary functions to perform
  this step;
\item
  analysis\_neural\_network.py: implement the MLP approach on the
  dataset;
\item
  analysis\_lightgbm.py: apply the LightGbm on the dataset;
\item
  compare\_predictions.R: contrasts the different predictive approaches
  on the company test data set,
\end{enumerate}

\hypertarget{data-preparation-and-anonymization}{%
\subsection{Data Preparation and
Anonymization}\label{data-preparation-and-anonymization}}

The market and company data file were loaded. An initial renaming of the
variable has been performed, conventionally naming the continuous one as
cont\_x while the categorical one as cat\_x, being \(x\) a number from
one up to the number of variables of such category. The following
criterion have been used to filter out anomalous observations: presence
of missing values in any of the observations, zero exposures.

Then, the available data has been split threefold: the last available
year has been set to the test set, while the remaining years have been
split into a train / validation set using a 80/20 ratio. Therefore we
have available three dataset for the marked data, and another three for
the company one.

\hypertarget{aknowledgments}{%
\subsection{Aknowledgments}\label{aknowledgments}}

\hypertarget{references}{%
\section*{References}\label{references}}
\addcontentsline{toc}{section}{References}

\hypertarget{refs}{}
\begin{CSLReferences}{1}{0}
\leavevmode\hypertarget{ref-antonio2007actuarial}{}%
Antonio, Katrien, and Jan Beirlant. 2007. {``Actuarial Statistics with
Generalized Linear Mixed Models.''} \emph{Insurance: Mathematics and
Economics} 40 (1): 58--76.

\leavevmode\hypertarget{ref-Xacur2018}{}%
{``{Bayesian credibility for GLMs}.''} 2018. \emph{Insurance:
Mathematics and Economics}.
\url{https://doi.org/10.1016/j.insmatheco.2018.05.001}.

\leavevmode\hypertarget{ref-bergstra2013making}{}%
Bergstra, James, Daniel Yamins, and David Cox. 2013. {``Making a Science
of Model Search: Hyperparameter Optimization in Hundreds of Dimensions
for Vision Architectures.''} In \emph{International Conference on
Machine Learning}, 115--23. PMLR.

\leavevmode\hypertarget{ref-chen2016xgboost}{}%
Chen, Tianqi, and Carlos Guestrin. 2016. {``Xgboost: A Scalable Tree
Boosting System.''} In \emph{Proceedings of the 22nd Acm Sigkdd
International Conference on Knowledge Discovery and Data Mining},
785--94.

\leavevmode\hypertarget{ref-Danzon1983}{}%
Danzon, Patricia Munch. 1983. {``{Rating Bureaus in U.S. Property
Liability Insurance Markets: Anti or Pro-competitive?}''} \emph{The
Geneva Papers on Risk and Insurance - Issues and Practice} 8 (4):
371--402. \url{https://doi.org/10.1057/gpp.1983.42}.

\leavevmode\hypertarget{ref-Elith2008}{}%
Elith, J., J. R. Leathwick, and T. Hastie. 2008. {``{A working guide to
boosted regression trees}.''}
\url{https://doi.org/10.1111/j.1365-2656.2008.01390.x}.

\leavevmode\hypertarget{ref-ferrario2020insights}{}%
Ferrario, Andrea, Alexander Noll, and Mario V Wuthrich. 2020.
{``Insights from Inside Neural Networks.''} \emph{Available at SSRN
3226852}.

\leavevmode\hypertarget{ref-giniFrees}{}%
Frees, Edward W. (Jed), Glenn Meyers, and A. David Cummings. 2014.
{``Insurance Ratemaking and a Gini Index.''} \emph{The Journal of Risk
and Insurance} 81 (2): 335--66.
\url{http://www.jstor.org/stable/24546807}.

\leavevmode\hypertarget{ref-goldburd2016generalized}{}%
Goldburd, Mark, Anand Khare, and Dan Tevet. 2016. \emph{{Generalized
Linear Models for Insurance Rating}}. 5.
\url{https://doi.org/10.2307/1270349}.

\leavevmode\hypertarget{ref-Goodfellow-et-al-2016}{}%
Goodfellow, Ian, Yoshua Bengio, and Aaron Courville. 2016. \emph{{Deep
Learning}}. Adaptive Computation and Machine Learning. MIT Press.

\leavevmode\hypertarget{ref-ke2017lightgbm}{}%
Ke, Guolin, Qi Meng, Thomas Finley, Taifeng Wang, Wei Chen, Weidong Ma,
Qiwei Ye, and Tie-Yan Liu. 2017. {``Lightgbm: A Highly Efficient
Gradient Boosting Decision Tree.''} \emph{Advances in Neural Information
Processing Systems} 30: 3146--54.

\leavevmode\hypertarget{ref-prokhorenkova2017catboost}{}%
Prokhorenkova, Liudmila, Gleb Gusev, Aleksandr Vorobev, Anna Veronika
Dorogush, and Andrey Gulin. 2017. {``CatBoost: Unbiased Boosting with
Categorical Features.''} \emph{arXiv Preprint arXiv:1706.09516}.

\leavevmode\hypertarget{ref-RSoftware}{}%
R Core Team. 2021. \emph{R: A Language and Environment for Statistical
Computing}. Vienna, Austria: R Foundation for Statistical Computing.
\url{https://www.R-project.org/}.

\leavevmode\hypertarget{ref-schelldorfer2019nesting}{}%
Schelldorfer, Jürg, and Mario V Wuthrich. 2019. {``Nesting Classical
Actuarial Models into Neural Networks.''} \emph{Available at SSRN
3320525}.

\leavevmode\hypertarget{ref-spedicato2018machine}{}%
Spedicato, Giorgio Alfredo, Christophe Dutang, and Leonardo Petrini.
2018. {``Machine Learning Methods to Perform Pricing Optimization. A
Comparison with Standard GLMs.''} \emph{Variance} 12 (1): 69--89.

\leavevmode\hypertarget{ref-python3}{}%
Van Rossum, Guido, and Fred L Drake. 2009. \emph{{Python 3 Reference
Manual}}. Scotts Valley, CA: CreateSpace.

\leavevmode\hypertarget{ref-yan_applications_2009}{}%
Yan, Jun, James Guszcza, Matthew Flynn, and Cheng-Sheng Peter Wu. 2009.
{``Applications of the Offset in Property-Casualty Predictive
Modeling.''} In \emph{Casualty {Actuarial} {Society} {E}-{Forum},
{Winter} 2009}, 366.

\end{CSLReferences}

\end{document}
