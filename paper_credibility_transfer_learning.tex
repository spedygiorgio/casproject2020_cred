% Options for packages loaded elsewhere
\PassOptionsToPackage{unicode}{hyperref}
\PassOptionsToPackage{hyphens}{url}
%
\documentclass[
]{article}
\usepackage{lmodern}
\usepackage{amsmath}
\usepackage{ifxetex,ifluatex}
\ifnum 0\ifxetex 1\fi\ifluatex 1\fi=0 % if pdftex
  \usepackage[T1]{fontenc}
  \usepackage[utf8]{inputenc}
  \usepackage{textcomp} % provide euro and other symbols
  \usepackage{amssymb}
\else % if luatex or xetex
  \usepackage{unicode-math}
  \defaultfontfeatures{Scale=MatchLowercase}
  \defaultfontfeatures[\rmfamily]{Ligatures=TeX,Scale=1}
\fi
% Use upquote if available, for straight quotes in verbatim environments
\IfFileExists{upquote.sty}{\usepackage{upquote}}{}
\IfFileExists{microtype.sty}{% use microtype if available
  \usepackage[]{microtype}
  \UseMicrotypeSet[protrusion]{basicmath} % disable protrusion for tt fonts
}{}
\makeatletter
\@ifundefined{KOMAClassName}{% if non-KOMA class
  \IfFileExists{parskip.sty}{%
    \usepackage{parskip}
  }{% else
    \setlength{\parindent}{0pt}
    \setlength{\parskip}{6pt plus 2pt minus 1pt}}
}{% if KOMA class
  \KOMAoptions{parskip=half}}
\makeatother
\usepackage{xcolor}
\IfFileExists{xurl.sty}{\usepackage{xurl}}{} % add URL line breaks if available
\IfFileExists{bookmark.sty}{\usepackage{bookmark}}{\usepackage{hyperref}}
\hypersetup{
  pdftitle={Adjusting Manual Rates to Own Experience: Comparing the Credibility Approach to Machine Learning},
  pdfauthor={Giorgio A. Spedicato, Ph.D FCAS FSA CSPA, Christophe Dutang, Ph.D, Quentin Guibert, Ph.D},
  hidelinks,
  pdfcreator={LaTeX via pandoc}}
\urlstyle{same} % disable monospaced font for URLs
\usepackage[margin=1in]{geometry}
\usepackage{longtable,booktabs}
\usepackage{calc} % for calculating minipage widths
% Correct order of tables after \paragraph or \subparagraph
\usepackage{etoolbox}
\makeatletter
\patchcmd\longtable{\par}{\if@noskipsec\mbox{}\fi\par}{}{}
\makeatother
% Allow footnotes in longtable head/foot
\IfFileExists{footnotehyper.sty}{\usepackage{footnotehyper}}{\usepackage{footnote}}
\makesavenoteenv{longtable}
\usepackage{graphicx}
\makeatletter
\def\maxwidth{\ifdim\Gin@nat@width>\linewidth\linewidth\else\Gin@nat@width\fi}
\def\maxheight{\ifdim\Gin@nat@height>\textheight\textheight\else\Gin@nat@height\fi}
\makeatother
% Scale images if necessary, so that they will not overflow the page
% margins by default, and it is still possible to overwrite the defaults
% using explicit options in \includegraphics[width, height, ...]{}
\setkeys{Gin}{width=\maxwidth,height=\maxheight,keepaspectratio}
% Set default figure placement to htbp
\makeatletter
\def\fps@figure{htbp}
\makeatother
\setlength{\emergencystretch}{3em} % prevent overfull lines
\providecommand{\tightlist}{%
  \setlength{\itemsep}{0pt}\setlength{\parskip}{0pt}}
\setcounter{secnumdepth}{-\maxdimen} % remove section numbering
\usepackage{draftwatermark}
\ifluatex
  \usepackage{selnolig}  % disable illegal ligatures
\fi
\newlength{\cslhangindent}
\setlength{\cslhangindent}{1.5em}
\newlength{\csllabelwidth}
\setlength{\csllabelwidth}{3em}
\newenvironment{CSLReferences}[2] % #1 hanging-ident, #2 entry spacing
 {% don't indent paragraphs
  \setlength{\parindent}{0pt}
  % turn on hanging indent if param 1 is 1
  \ifodd #1 \everypar{\setlength{\hangindent}{\cslhangindent}}\ignorespaces\fi
  % set entry spacing
  \ifnum #2 > 0
  \setlength{\parskip}{#2\baselineskip}
  \fi
 }%
 {}
\usepackage{calc}
\newcommand{\CSLBlock}[1]{#1\hfill\break}
\newcommand{\CSLLeftMargin}[1]{\parbox[t]{\csllabelwidth}{#1}}
\newcommand{\CSLRightInline}[1]{\parbox[t]{\linewidth - \csllabelwidth}{#1}\break}
\newcommand{\CSLIndent}[1]{\hspace{\cslhangindent}#1}

\title{Adjusting Manual Rates to Own Experience: Comparing the
Credibility Approach to Machine Learning}
\author{Giorgio A. Spedicato, Ph.D FCAS FSA CSPA, Christophe Dutang,
Ph.D, Quentin Guibert, Ph.D}
\date{26 giugno, 2021}

\begin{document}
\maketitle

\hypertarget{introduction}{%
\section{Introduction}\label{introduction}}

The use of market data as an aid for setting own company rates has been
a common practice in the Insurance Industry. External data, as provided
by Insurance Rating Bureaus or Advisory Organizations, may supplement
internal company's ones that may be scarce, unreliable because of a
non-representative and/or a too short history, or, non-existing at all,
e.g.~when entering in a new business lines or territory. The importance
of external data to both support adequate rates that preserve company
solvency and in easing the entrance of new players has been historically
recognized by regulator, e.g.~granting a partial antitrust -- law
exception in the US jurisdiction (Danzon 1983).

When the Insurer takes into account its own experience in order to
enhance the credibility of its rates, it need to assess if and in which
manner its portfolio experience departs from the market one. The
actuarial profession traditionally used techniques based on Bayesian
statistics and non-parametric credibility to optimally combine the
market and insurer's portfolio experience in the technical rates. These
early models were not based on policyholders' rate-making variables, yet
some advanced regression credibility models have been proposed in the
actuarial literature such as the Hachemeister model, see e.g. (Bühlmann
and Gisler 2006). On the contrary, rates based on Generalized Linear
Models (GLM), the current gold standard in personal rates pricing
(Goldburd, Khare, and Tevet 2016), are only based on the impact of
ratemaking factors, giving no credit to the individual policy
experience. Nevertheless, mixed effects GLMs allow to incorporate
policyholders' experience within GLM tariff structure ({``{Bayesian
credibility for GLMs}''} 2018; Antonio and Beirlant 2007) but they are
not widespread used.

The recent widespread/massive usage of Machine Learning has provided
many more techniques to the practitioner actuaries' toolset. In
particular, the Gradient Boosting Models (GBM) and the Deep Learning
(DL) frameworks can be used in a manner that permits to ``transfer''
what the model has learnt on a much bigger data set (as the market data,
MMT) to a smaller set (the portfolio data of the company, CPN).
``Transfer learning'' (henceforth TFL) is typically in computer vision
DL modes to fine tune standard architecture on specific recognition
tasks. An ``initial score'' can be provided to GBMs to take into account
the known effect of exposure or a-priori modeled estimate before
``boosting'' the prediction. As ML can easily handle complex non --
linear relationship, it is reasonable to believe that they could assess
the policyholders' risk more precisely than standard credibility or GLMs
based approaches. The proposed research aims to contrast traditional
methods to ML ones in the task of blending market data to individual
portfolio experience. After a brief business and methodological
introduction, we will apply on a (properly anonymized) data set
comprised by both market and own portfolio experience relative to a
European country non -- life business line. The final comparisons will
consider not only on the predictive performance, but also in the ease of
practical application in term of computational request, ease of
understanding and interpretability of the results.

\hypertarget{overview-of-the-use-of-external-experience-in-own-business-in-the-insurance-industry}{%
\section{Overview of the use of external experience in own business in
the insurance
industry}\label{overview-of-the-use-of-external-experience-in-own-business-in-the-insurance-industry}}

To be discussed:

\begin{itemize}
\tightlist
\item
  historical practice in the US; current use of insurance bureau;
\item
  use of external experience, credibility
\item
  Role of reinsurance as providing advice and support in pricing.
\end{itemize}

\hypertarget{modeling-approach}{%
\section{Modeling approach}\label{modeling-approach}}

The research aims to compare the predictive power of traditional and
machine learning methods that use an initial estimate of loss costs,
e.g.~from MTK experience, to predict those of a smaller portion (the CPN
one) in a subsequent period (the test set). Therefore, the modeling
process aims to predict the losses on the last available year (the test
set) by training models on the experience of the previous years,
eventually split into a train and validation test.

The empirical data available for the study regards a risk for which year
to year loss cost may materially fluctuate due to external condition
(systematic variability) much more than the portolios' risks
heterogeneity composition. At this regards, the performance assessment
has considered not only the discrepancy between the actual and predicted
losses, but the ability of the model to rank risks.

The losses are the number of damaged units while the exposure are the
number of insured units. Therefore only the frequency component has been
modeling, choosing either a binomial or a Poisson loss function.
Henceforth losses in this paper shall be considered as synonym of claim
number.

We are using models that permits to use an initial estimate of losses
performed on another set (transfer learning). While the paper explores
the use of such approach applying ML methods, traditional GLM may be
used as well. E.g. under a log-linear regression framework and initial
log-estimate of either the frequency, the severity of the pure - premium
may be set as an offset (Yan et al. 2009) for a subsequent model.

\hypertarget{machine-learning-techniques}{%
\subsection{Machine learning
techniques}\label{machine-learning-techniques}}

ML methods have been acquiring increasing attention by actuarial
practitioners especially. Beginning from the analysis of policyholders'
behavior (Spedicato, Dutang, and Petrini 2018), several applications
have sprung also for risk pricing. An application of boosting techniques
to estimate the frequency and the severity of an MTPL dataset can be
found in Noll (Noll, Salzmann, and Wuthrich 2020), while Schelldorfer
and Wutrict presented a joint model that boosts GLM performance using
Deep Learning (Schelldorfer and Wuthrich 2019). ML methods used in
insurance pricing are strongly non - linear and are able to
automatically find interactions among ratemaking factors and exclude non
relevant features. In particular two techniques are acquiring widespread
importance: Boosting and Deep learning. Both techniques allow the use of
an initial estimate of loss / exposure to risk tho be feed to the model
to be fine tuned for the current dataset.

When applied to claim frequency prediction, ML models are fit to
minimize a Poisson log-loss function. In addition, to handle uneven risk
exposure, the log - measure of exposure risk is given (in log scale) as
an init-score (\(F_t\left(x\right)\)) to initialize the learning
process. The init-score (or base margin) in the boosting approach has
the same role of the traditional GLM offset term (Goldburd, Khare, and
Tevet 2016). As lately explained, the offset - init score is used in the
TRF context to provide the initial estimate of the loss / risk measure
based on MKT data.

\hypertarget{brief-overview}{%
\subsection{Brief overview}\label{brief-overview}}

\hypertarget{boosting-techniques}{%
\subsection{Boosting techniques}\label{boosting-techniques}}

The boosting approach can be synthesized by the following relation:

\[F_t\left(x\right)=F_{t-1}\left(x\right)+\eta*h_t\left(x\right)\] that
is, the prediction at the \(t\)-it step is given by the contribution, to
the prediction of the previous step, of a weak predictor
\(h_t\left(x\right)\), properly weighted by a learning factor \(\eta\),
being \(x\) the covariate vector. The most common choice for the weak
predictor \(h_t\left(x\right)\) lies in the classification and
regression trees family, from which the Gradient Boosted Tree (GBT)
represent the most relevant models. It can be shown that ``boosting''
weak predictors lead to very strong predictive models (Elith, Leathwick,
and Hastie 2008). Almost all winning solutions of data science
competitions held by Kaggle are at least partially based on XGBoost
(Chen and Guestrin 2016), the most famous GBT model. More recent and
interesting alternatives to be tested are: LightGbm (Ke et al. 2017),
which is particularly renowned for its speed, and Catboost
(Prokhorenkova et al. 2017), which has introduced an efficient solution
for handling categorical data.

A set of hyperparameters defines a boosted model, and even more define a
GBT one. The core hyperparameters that influence the boosting part are:
the number of models (trees), \(t=1,2,\ldots, T\) (typically between 100
and 1000) and the learning rate \(\eta\), whose typical values lies
between 0.2 and 0.001. \(h_t\left(x\right)\) can be, when it belongs to
the CART family, the maximum depth, the minimum number of observation in
final leafs, the fraction of observation (rows or columns) that are
considered when growing each tree. The optimal combination of
hyperparameters is learn using either a grid search approach or a more
refined one (e.g.~bayesian optimization).

\hypertarget{deep-learning}{%
\subsection{Deep Learning}\label{deep-learning}}

An artificial network is a mathematical structure that applies a non
linear function to a linear combination of inputs, say
\(\phi\left(\bar x_i^T*\bar w+\beta\right)\), being \(\bar w\) and
\(\beta\) the weights and bias respectively. A neural network consists
in one o more layer of interconnected neurons, that receives a (possibly
multivariate) input set and retrieves and output set. Modern Deep Neural
Networks are constructed by many (deep) layers of neurons. Deep Learning
has been knowing a hype in interest for a decade, thanks to the
availability of huge amount of data, computing power (in particular GPU
computing) and the development of newer approaches to reduce the
overfitting that had halted the widespread adoption of such techniques
in previous decades (Goodfellow, Bengio, and Courville 2016). Different
architectures has reached state of the art performances in many fields;
e.g.~convolutionary neural networks achieved top performance in computer
vision (e.g.~image classification and object detection), while recurrent
neural networks (e.g.~Long Short Term Memory ones) provides excellent
results in Natural language processing tasks like sequence-to-sequence
modeling (translation) and text classification (sentiment analysis).

Simpler structure are needed for a claim frequency regression, the
multi-layer perception (MLP) architectures that basically consist in
stacked simple neurons layers, from the input one to the single output
cell one. This structure is dealt to handle the relation between the
relation between the ratemaking factors and the frequency (the
structural part). To handle the different exposures, the proposed
architecture is based on the solution presented by (Ferrario, Noll, and
Wuthrich 2020; Schelldorfer and Wuthrich 2019). A separate branch
collects the exposure, applies a log-transformation, then this exposure
is added in a specific layer just before the final one (that has a
dimension of one).

Training a DL model consists in providing batches of data to the
network, evaluate the loss performance and updating the weights in the
direction that minimize the training (backpropagation). The whole data
set is provided to the fitting algorithms many times (epochs) split in
batch. One of the common practice to avoid overfitting is to use a
validation set where the loss is scored at each epoch. When it starts to
systematically diverge, the training process is stopped (early
stopping).

\emph{TODO}

\hypertarget{credibility-based-models}{%
\section{Credibility-based models}\label{credibility-based-models}}

\emph{TODO}

\hypertarget{notation}{%
\subsection{Notation}\label{notation}}

\hypertarget{buxfchlmann-straub-credibility-model}{%
\subsection{Bühlmann-Straub Credibility
Model}\label{buxfchlmann-straub-credibility-model}}

\hypertarget{application}{%
\section{Application}\label{application}}

The analysis was performed on the two data sets, the CMP and MKT,
preprocessed and split into train, validation and test set as previously
discussed. Then, the models were fit on the train set and the predictive
performance was assessed on the test set. The validation set was used in
DL and BST models to avoid overfitting.

Finally, the models are compared in terms of predictive accuracy, using
the (The actual / predicted ratio) and risk classification performance,
using the Normalized Gini index (Frees, Meyers, and Cummings 2014). The
latter index has become quite popular in the actuarial academia and
practictioners to compare competing risk models.

\hypertarget{the-structure-of-the-dataset}{%
\subsection{The structure of the
dataset}\label{the-structure-of-the-dataset}}

Two (anonymized) dataset were provided, one for the marketwide
(\texttt{"mkt\_anonymized\_data.csv"}) and one for the company
(\texttt{"mkt\_anonumized\_data.csv"}), henceforth MKT and CMP datasets.
The datasets share the same structure, as each company provides its data
in the same format to the Pool, that aggregates individual filings into
a marketwide file, that is provided back to the companies. The dataset
contains exposures and claims aggregated by some classification
variables. Variable names, levels and numeric variable distribution have
been masked and anonymized for privacy and confidentiality purposes.

\textbf{maybe a graphic could be done?}

The following variables are contained in the provided data set:

\begin{itemize}
\tightlist
\item
  \emph{exposure}: the insurance exposure measures by classification
  group, on which the rate is filled (aggregated outcomes);
\item
  \emph{claims}: the number of claims by classification group
  (aggregated outcomes);
\item
  \emph{ID}: unique row number (helper variable);
\item
  \emph{zone\_id}: territory (aggregating variable);
\item
  \emph{year}: filing year (aggregating variable);
\item
  \emph{group}: random partition of the dataset into train, valid and
  test set.
\item
  \emph{cat1}: categorical variable 1, available in the original file
  (aggregating variable);
\item
  \emph{cat2}: categorical variable 2, available in the original file
  (aggregating variable);
\item
  \emph{cat3}: categorical variable 3, available in the original file
  (aggregating variable);
\item
  \emph{cat4-cat8}: categorical variables related to the territory
  (joined to the original file by zone\_id);
\item
  \emph{cont1-cont12}: numeric variables related to the territory
  (joined to the original file by zone\_id);
\end{itemize}

Categorical and continuous variables have been anonymized by label
encoding and scaling (calibrated on market data). In addition, the last
available year (2008) has been used as test set, while data from
precedent years have been randomly split between train and validation
sets on a 80/20 basis.

Market data is available for 11 years, while company data for the last
five one. Also, the number of exposures is widely dependent by the cat1
variable.

\hypertarget{ml-techniques-implementation}{%
\subsection{ML techniques
implementation}\label{ml-techniques-implementation}}

\hypertarget{boosting-approach}{%
\subsubsection{Boosting approach}\label{boosting-approach}}

The Lightgbm model has been used to apply boosted trees on the provided
data sets, minimizing Poisson deviance. As for most modern ML methods, a
lightgbm model is fully defined by a set of many hyperparameter for
which default values may not be optimal for the given data and there is
no closed formula to identify the best combination for the given data.

Therefore an iperparameter optimization step was performed. For each
hyperparameter a range of variation was set, then a 100-run trial was
performed using a Bayesian Optimization approach performed by the
hyperopt python library (Bergstra, Yamins, and Cox 2013). Under the BO
approach, each subsequent iteration is perfomed toward the point that
minimize the loss to be optimized, being the loss distribution by
hyperparameter updated each iteration using a bayesian approach.

As suggested by boosting trees practitioners, the number of boosted
models was not estimated under the BO approach but determined by early
stopping. The loss was scored under the validation set and the number of
trees chosen is that beyond which the loss stop to decrease and start
diverging up.

The CMP and MKT model used the standard exposure (in logarithm base) as
init score. The TRF model instead uses as init score the ``a-priori''
prediction of the MKT model on the CMP data.

\hypertarget{deep-learning-1}{%
\subsubsection{Deep Learning}\label{deep-learning-1}}

The chose DL architecture was set by several trials, based on previous
experiments and symilar architecture found in the literature for tabular
data analysis. Unfortunately, the hyperparameter space of a DL
architecture is very vaste, comprising not only fitting level degrees of
freedom (the optymizer, the number of epochs, the batch size) but the
whole layers' architecture: the number of layers, the numbers of neuron
within etc\ldots{} At this regard it is common among practictioners to
starts with a knowling working architecture in a symilar field and
perform moderate changes. While more systemating DL architecture
optimization approaches are being developed (e.g.~the Neural
Architecture Search) the use of such techniques was out of the scope of
the paper

\begin{figure}
\centering
\includegraphics{./materials/dl_model.png}
\caption{DL model structure}
\end{figure}

The same model shown above was used for both the CMP, MKT and TRF
models. A dense layer collects the inputs, where the categorical
variables have been handled using embedding. Three hidden layers perform
the feature engineering and knowledge extraction from the input; Dropout
layers have been added to robustify the process. As anticipated in the
methodological section, the exposure part is separately handled in
another branch and then merged in the final layer.

Overfitting was controlled using an EarlyStopping callback scoring the
loss on the validation test and stop learning over next epochs if the
loss did not improve for more than 20 epochs.

The TRF model has been build using the pre-trained weights calculated on
the market data and continuing the traiing process on the CPN data.

\hypertarget{credibility}{%
\subsection{Credibility}\label{credibility}}

\hypertarget{comparison}{%
\subsection{Comparison}\label{comparison}}

The table below reports the predictive performance, evaluated on the
company test set, for the DL and Bst models' families. The columns
approach indicates whether the model was trained on market-only (mkt),
company-only (cpy) or company data using a transfer learning approach
(trf).

\begin{longtable}[]{@{}llrr@{}}
\caption{Models comparison}\tabularnewline
\toprule
model & approach & normalized\_gini &
actual\_predicted\_ratio\tabularnewline
\midrule
\endfirsthead
\toprule
model & approach & normalized\_gini &
actual\_predicted\_ratio\tabularnewline
\midrule
\endhead
dl & mkt & 0.921 & 0.924\tabularnewline
dl & cpn & 0.909 & 0.775\tabularnewline
dl & trf & 0.925 & 0.967\tabularnewline
bst & mkt & 0.939 & 0.975\tabularnewline
bst & cpn & 0.924 & 0.841\tabularnewline
bst & trf & 0.940 & 1.052\tabularnewline
\bottomrule
\end{longtable}

First, we see that the actual/predicted ratio is between 0.9 - 1.1 for
all models, but company ones where it is somewhat worse. We anticipate
that as the test set considers a year different from the train and
validation pool, the predictions may be structurally biased as the
insured risk depends by the year's climate and that frequency trending
is not consider in the modeling framework at all. The transfer learning
approach offers higher predictive accuracy when measured by the NG index
and the predictive performance is the highest among all competitive
approaches except for the boosting approach. Regarding the predictive
accuracy, on the other hand, and expecially for DL models, we cannot
rule out that the superiority of TFR approaches holds for all possible
MLP architectures. Also, it is likely to happen that as far as the
company data increases, the advantage of the TRF approach decreases with
respect to a model trained on company data only.

\emph{TODO}

\hypertarget{conclusion}{%
\section{Conclusion}\label{conclusion}}

We presented and application of TF that can be resembled to the
traditional ``credibility'' approach to transfer the experience applied
on a different, but similar, book of business to a newer one. We saw
that ML approach may provides interesting results and may be worth to
try with.

Finally, we performed our empirical analysis transferring loss
experience from an external insurance bureau to a specific company
portfoglio. This ``transfer of experience'' may be also performed within
the same company for example when new products, taylored for niche books
of business, are created. Initial losses estimates may be performed on
the initial product and then applied as initial scores on the newer
portfolio.

\emph{TODO}

\hypertarget{appendix}{%
\section{Appendix}\label{appendix}}

\hypertarget{code}{%
\subsection{Code}\label{code}}

The modeling has been performed using both R (R Core Team 2021) and (Van
Rossum and Drake 2009). The following files has been provided:

\begin{enumerate}
\def\labelenumi{\arabic{enumi}.}
\setcounter{enumi}{-1}
\item
  preprocess and anonymize.py: work on original data, anonymizing column
  and internal datasets and split data across train, validation and test
  set; the config\_all.py file provides ancillary functions to perform
  this step;
\item
  analysis\_neural\_network.py: implement the MLP approach on the
  dataset;
\item
  analysis\_lightgbm.py: apply the LightGbm on the dataset;
\item
  compare\_predictions.R: contrasts the different predictive approaches
  on the company test data set,
\end{enumerate}

\hypertarget{data-preparation-and-anonymization}{%
\subsection{Data Preparation and
Anonymization}\label{data-preparation-and-anonymization}}

The market and company data file were loaded. An initial renaming of the
variable has been performed, conventionally naming the continuous one as
cont\_x while the categorical one as cat\_x, being \(x\) a number from
one up to the number of variables of such category. The following
criterion have been used to filter out anomalous observations: presence
of missing values in any of the observations, zero exposures.

Then, the available data has been split threefold: the last available
year has been set to the test set, while the remaining years have been
split into a train / validation set using a 80/20 ratio. Therefore we
have available three dataset for the marked data, and another three for
the company one.

\hypertarget{aknowledgments}{%
\subsection{Aknowledgments}\label{aknowledgments}}

\hypertarget{references}{%
\section*{References}\label{references}}
\addcontentsline{toc}{section}{References}

\hypertarget{refs}{}
\begin{CSLReferences}{1}{0}
\leavevmode\hypertarget{ref-antonio2007actuarial}{}%
Antonio, Katrien, and Jan Beirlant. 2007. {``Actuarial Statistics with
Generalized Linear Mixed Models.''} \emph{Insurance: Mathematics and
Economics} 40 (1): 58--76.

\leavevmode\hypertarget{ref-Xacur2018}{}%
{``{Bayesian credibility for GLMs}.''} 2018. \emph{Insurance:
Mathematics and Economics}.
\url{https://doi.org/10.1016/j.insmatheco.2018.05.001}.

\leavevmode\hypertarget{ref-bergstra2013making}{}%
Bergstra, James, Daniel Yamins, and David Cox. 2013. {``Making a Science
of Model Search: Hyperparameter Optimization in Hundreds of Dimensions
for Vision Architectures.''} In \emph{International Conference on
Machine Learning}, 115--23. PMLR.

\leavevmode\hypertarget{ref-buhlmann2006course}{}%
Bühlmann, Hans, and Alois Gisler. 2006. \emph{A Course in Credibility
Theory and Its Applications}. Springer Science \& Business Media.

\leavevmode\hypertarget{ref-chen2016xgboost}{}%
Chen, Tianqi, and Carlos Guestrin. 2016. {``Xgboost: A Scalable Tree
Boosting System.''} In \emph{Proceedings of the 22nd Acm Sigkdd
International Conference on Knowledge Discovery and Data Mining},
785--94.

\leavevmode\hypertarget{ref-Danzon1983}{}%
Danzon, Patricia Munch. 1983. {``{Rating Bureaus in U.S. Property
Liability Insurance Markets: Anti or Pro-competitive?}''} \emph{The
Geneva Papers on Risk and Insurance - Issues and Practice} 8 (4):
371--402. \url{https://doi.org/10.1057/gpp.1983.42}.

\leavevmode\hypertarget{ref-Elith2008}{}%
Elith, J., J. R. Leathwick, and T. Hastie. 2008. {``{A working guide to
boosted regression trees}.''}
\url{https://doi.org/10.1111/j.1365-2656.2008.01390.x}.

\leavevmode\hypertarget{ref-ferrario2020insights}{}%
Ferrario, Andrea, Alexander Noll, and Mario V Wuthrich. 2020.
{``Insights from Inside Neural Networks.''} \emph{Available at SSRN
3226852}.

\leavevmode\hypertarget{ref-giniFrees}{}%
Frees, Edward W. (Jed), Glenn Meyers, and A. David Cummings. 2014.
{``Insurance Ratemaking and a Gini Index.''} \emph{The Journal of Risk
and Insurance} 81 (2): 335--66.
\url{http://www.jstor.org/stable/24546807}.

\leavevmode\hypertarget{ref-goldburd2016generalized}{}%
Goldburd, Mark, Anand Khare, and Dan Tevet. 2016. \emph{{Generalized
Linear Models for Insurance Rating}}. 5.
\url{https://doi.org/10.2307/1270349}.

\leavevmode\hypertarget{ref-goldburd2016generalized}{}%
---------. 2016. \emph{{Generalized Linear Models for Insurance
Rating}}. 5. \url{https://doi.org/10.2307/1270349}.

\leavevmode\hypertarget{ref-Goodfellow-et-al-2016}{}%
Goodfellow, Ian, Yoshua Bengio, and Aaron Courville. 2016. \emph{{Deep
Learning}}. Adaptive Computation and Machine Learning. MIT Press.

\leavevmode\hypertarget{ref-ke2017lightgbm}{}%
Ke, Guolin, Qi Meng, Thomas Finley, Taifeng Wang, Wei Chen, Weidong Ma,
Qiwei Ye, and Tie-Yan Liu. 2017. {``Lightgbm: A Highly Efficient
Gradient Boosting Decision Tree.''} \emph{Advances in Neural Information
Processing Systems} 30: 3146--54.

\leavevmode\hypertarget{ref-noll2020case}{}%
Noll, Alexander, Robert Salzmann, and Mario V Wuthrich. 2020. {``Case
Study: French Motor Third-Party Liability Claims.''} \emph{Available at
SSRN 3164764}.

\leavevmode\hypertarget{ref-prokhorenkova2017catboost}{}%
Prokhorenkova, Liudmila, Gleb Gusev, Aleksandr Vorobev, Anna Veronika
Dorogush, and Andrey Gulin. 2017. {``CatBoost: Unbiased Boosting with
Categorical Features.''} \emph{arXiv Preprint arXiv:1706.09516}.

\leavevmode\hypertarget{ref-RSoftware}{}%
R Core Team. 2021. \emph{R: A Language and Environment for Statistical
Computing}. Vienna, Austria: R Foundation for Statistical Computing.
\url{https://www.R-project.org/}.

\leavevmode\hypertarget{ref-schelldorfer2019nesting}{}%
Schelldorfer, Jürg, and Mario V Wuthrich. 2019. {``Nesting Classical
Actuarial Models into Neural Networks.''} \emph{Available at SSRN
3320525}.

\leavevmode\hypertarget{ref-spedicato2018machine}{}%
Spedicato, Giorgio Alfredo, Christophe Dutang, and Leonardo Petrini.
2018. {``Machine Learning Methods to Perform Pricing Optimization. A
Comparison with Standard GLMs.''} \emph{Variance} 12 (1): 69--89.

\leavevmode\hypertarget{ref-spedicato2018machine}{}%
---------. 2018. {``Machine Learning Methods to Perform Pricing
Optimization. A Comparison with Standard GLMs.''} \emph{Variance} 12
(1): 69--89.

\leavevmode\hypertarget{ref-python3}{}%
Van Rossum, Guido, and Fred L Drake. 2009. \emph{{Python 3 Reference
Manual}}. Scotts Valley, CA: CreateSpace.

\leavevmode\hypertarget{ref-yan_applications_2009}{}%
Yan, Jun, James Guszcza, Matthew Flynn, and Cheng-Sheng Peter Wu. 2009.
{``Applications of the Offset in Property-Casualty Predictive
Modeling.''} In \emph{Casualty {Actuarial} {Society} {E}-{Forum},
{Winter} 2009}, 366.

\end{CSLReferences}

\end{document}
